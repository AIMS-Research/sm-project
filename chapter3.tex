\chapter{Third Chapter}

Theorems before the chapter's first section will be dot-zero, 
and their numbering is completely wrong. You can avoid this
by simply always starting a chapter with a section. Ta Da! 
It will probably help you structure it anyway. 

\begin{thm}[My Theorem2]
This is my theorem2.
\end{thm}
\begin{proof}
And it has no proof2.
\end{proof}

Explore and evaluate your model.

Text text text text text text text text text text text text text text
text text text text text text text text text text text text text text
text text text text text text text text text text text text text text
text text text text text text text text text text text text text text
text text text text text.

\section{This is a Section}
Text text text text text text text text text text text text text text
text text text text text text text text text text text text text text

\begin{thm}[My Theorem2]
This is my theorem2.
\end{thm}
\begin{proof}
And it has no proof2.
\end{proof}

text text text text text text text text text text text text text text
text text text text text text text text text text text text text text
text text text text text. 

\subsection{This is a Subsection}
Text text text text text text text text text text text text text text
text text text text text text text text text text text text text text
text text text text text text text text text text text text text text

\subsection{ABc}
text text text text text text text text text text text text text text
text text text text text. 

\subsection{This is a Subsection}
Text text text text text text text text text text text text text text
text text text text text text text text text text text text text text
text text text text text text text text text text text text text text

\begin{lem}
 My lemma
\end{lem}

\begin{itemize}
 \item{an list not numbered}
 \item{another bullet}
\end{itemize}

\begin{lem}
My other lemma.
\end{lem}

\section{This is a Section}
Text text text text text text text text text text text text text text
text text text
\begin{align} % do not use eqnarray. 
\label{2ya}
x & = y + y\\
\label{2yb}
& = 2y
\end{align}
see equations \ref{2ya} and \ref{2yb}
text text text text text text text text text text text text text text
text text text text text text text text text text text text text text
text text text text text. Now since you remember all those equation 
numbering methods in the Short Math Guide, you can easily do all kinds
of funky numbering, split numbering, aligned numbering, and combined
numbering. Just don't lose your head.

\section{More}

\begin{conj}
 A conjecture
\end{conj}

And an example too

\begin{exa}
 As if this were an example.
\end{exa}


